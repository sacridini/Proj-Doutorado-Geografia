\documentclass[twocolumn]{article}
\usepackage[portuguese]{babel}
\usepackage[utf8]{inputenc}
\usepackage{comment}
\usepackage{lipsum} 

% Keywords command
\providecommand{\keywords}[1]
{
  \small	
  \textbf{\text{Palavras-chave---}} #1
}

\title{Perspectivas sobre a Utilização de Algoritmos para o Processamento de Séries Temporais no Estudo de Distúrbios Florestais}
\author{Eduardo Ribeiro Lacerda }
\date{} %\the\year}

\begin{document}
\maketitle

\begin{comment}
    \begin{abstract}
        O objetivo deste trabalho é apresentar o estado da arte sobre a aplicação de técnicas de análise e séries temporais em imagens de satélite orbitais para o estudo de áreas de floresta que sofreram ou não algum tipo de distúrbio.    
    \end{abstract}
\end{comment}

\vspace{1cm}
\keywords{Séries Temporais; Algoritmos; Landtrendr; Distúrbios}

\section{Introdução}

Durante muito tempo, a incorporação do tempo como variável chave de estudos geográficos, principalmente os pautados na análise sistêmica da paisagem se deu de forma bastante limitada principalmente pela dificuldade de acesso a séries históricas confiáveis, assim como pela dificuldade de manipulação das mesmas, quase sempre em ambiente computacional limitado.
\par
Com o aumento significativo do uso da internet e da capacidade computacional das últimas décadas, a utilização de métodos computacionais na análise de dados científicos se tornaram cada vez mais presentes com uma maior possibilidade de acesso não só a grandes bases de dados como também a ambientes computacionais de alto desempenho. Essa realidade passou a ser possível para um público que antes dependia quase que exclusivamente das estruturas disponíveis mais próximas e imediatas. Isso vem representando uma mudança ainda mais significativa para os pesquisadores de países onde o acesso a estruturas científicas quase sempre se deu de forma mais limitada. Consequentemente, isso se traduz em possibilidades de estudos não só mais detalhados e estruturados como também de pesquisas com outras escalas de análise, o que por sua vez possibilita uma maior relevância científica internacional e maior reprodutibilidade dos estudos.
\par
A aplicação de técnicas de análise de séries temporais para o monitoramento espaço-temporal da paisagem tem surgido com maior força nos últimos anos como consequência dessa evolução histórica. A aplicação dessas técnicas é da mais diversa possível, vindo inicialmente quase que exclusivamente dos estudos econométricos e de poucas outras áreas científicas para a sua massificação em muitas outras áreas. Uma delas, e que será abordada neste trabalho é a utilização de técnicas de análise de séries temporais em imagens digitais orbitais com o intuito do monitoramento de áreas naturais florestadas, principalmente na detecção de distúrbios florestais. Esse tipo de aplicação vem ganhando força não só pelo entendimento das mudanças da paisagem e suas dinâmicas, como também pela preocupação com o monitoramento do desmatamento, restauração e regeneração natural que vem acontecendo na mesma paisagem. Esse monitoramento nunca teve tanta importância como hoje devido ao maior entendimento e comprometimento internacional em relação as variáveis influenciadoras das mudanças climáticas através de tratados e acordos de estimulo à conservação e restauração de áreas naturais. 
\par
Políticas de REDD+, assim como vários esforços de cooperação internacional como os propostos pelo Bonn Challenge e os Aichi Targets deficidos pela CDB (\textit{Convention on Biological Diversity}), tem como ferramenta principal de monitoramento técnicas de sensoriamento remoto e análise espacial fortemente vinculadas a ideia de entendimento do comportamento da paisagem ao longo do tempo ~\cite{BOS2019295, CROUZEILLES2019}. Estudos mais recentes vem demonstrando que os processos de degradação em florestas tropicais tem impacto similar ou até mesmo maior em relação as emissões de carbono que o desmatamento das mesmas ~\cite{Harris1573, Houghton2012, Grace2014}. É importante notar aqui que a utilização do termo distúrbio é diferente do de degradação. Enquanto distúrbio está associado a um único evento que pode ser tanto natural como antrópico, degradação se associa a um processo temporalmente maior de perda de biomassa e a um tipo de influência necessariamente antrópica se considerarmos a definição do IPCC (\textit{Intergovernmental Panel on Climate Change}) ou não necessariamente antrópica caso se considere a definição da FAO (\textit{Food and Agriculture Organization of the United Nations}) ~\cite{Hirschmugl2017}. 
\par
Além disso, outros estudos demonstram que as florestas tropicais tem sofrido mais pressões do que as temperadas. Um terço das florestas tropicais já foram perdidas para o desmatamento, e da área restante, 46\% da área é fragmentada, 30\% degradada e apenas 24\% ainda permanece em estado mais preservado. Sabendo disso, fica ainda mais clara a necessidade de entender quais as melhores técnicas que podem ser utilizadas para o monitoramento destes processos ~\cite{Hirschmugl2017}. 
\par
O momento político e histórico de acordo com a perspectiva da preservação do meio ambiente na qual estamos inseridos necessita ainda mais que as pesquisas de monitoramento de distúrbios sejam feitos de forma cada vez mais transparente e acessíveis a todos para que processos de degradação sejam detectados a tempo. Com isso, é importante entendermos quais tecnologias estão disponíveis e quais as possibilidades de monitoramento são possíveis, tendo como objetivo principal entender não só suas potencialidades como suas limitações. 
\par
Sendo assim, este trabalho tem como objetivo a apresentação e análise das características dos principais algoritmos de detecção de distúrbios, principalmente os especializados na detecção de mudanças em áreas florestadas, seja na perda como no ganho de biomassa.
\par

\section{Breve Histórico}

Estudos espaço temporais com o objetivo de detecção de mudanças não são novos na área do sensoriamento remoto. A clássica análise bi-temporal de dados previamente classificados não é nova mas ainda é presente em muitos estudos atuais. No entanto, esse tipo de estudo tende a conter um número maior de erros, já que quanto maior o número de imagens analisadas, maior a quantidade de mapas com erros associados que serão levados em consideração. De qualquer forma, este tipo de estudo tendem a ser o que possui o menor requisito de poder computacional, já que trata somente de processos de álgebra de bandas.
\par
Com o tempo, o maior poder computacional disponível não só para o processamento como para o armazenamento de dados possibilitou que outras técnicas mais elaboradas pudessem ser implementadas, onde ao invés de apenas algumas imagens serem consideradas, todas as imagens da série temporal são levadas em consideração. Isso possibilitou também que muitas técnicas de análise de séries temporais tipicamente aplicadas principalmente na área de econometria pudessem ser utilizadas em estudos geoespaciais.
\par
O processamento de imagens de satélite que historicamente sempre foi feito através do processamento de pixels individuais, passou a partir da década de 00 a ser feita em muitos casos através do delimitação de objetos com a popularização das imagens de alta-resolução espacial e consequentemente das técnicas de GEOBIA (\textit{Geographic Object-Based Image Analysis}). As análises pixel à pixel que pareciam estar cada vez decadentes, acabaram ressuscitando nos últimos anos devido a capacidade de processamento de séries temporais densas. A densidade da série é um ponto chave para esse retorno, já que para que a análise temporal seja bem sucedida e para que os algoritmos aplicados possam entender ainda melhor os processos ocorridos no tempo, é necessário que os dados utilizados tenham resolução temporal condizente com o que se quer detectar. No caso da utilização de imagens de satélite isso se torna um limitador importante, já que a disponibilidade de satélites imageadores historicamente nunca foi alta e tem custo extremamente elevado de produção quando comparado a outros tipos de sensores. Outro problema é que os satélites possuem um tempo de revisita que em muitos casos não possibilitam que uma série mais densa possa ser estruturada. 
%\par
Além disso, problemas como a presença de nuvens e sombras, assim como ruídos na própria imagem e a diferenciação espacial da área de imageamento do satélite devido a heterogeneidade da distribuição solar no planeta dificultam ainda mais esse processo. No entanto, com o tempo, outros satélites foram sendo desenvolvidos e propositalmente pensados com o objetivo de incorporar as séries de imagens já existentes derivadas de projetos antigos e abandonados com os recém lançados. Um exemplo disso foi o lançamento da série Sentinel 2 (A e B) pela Agência Espacial Europeia com uma resolução espectral e espacial similar as encontradas na série Landsat, o que possibilita uma fusão entre imagens de satélites diferentes com o objetivo de aumentar a densidade de imagens. Com isso, tanto a disponibilidade de dados como a resolução temporal dos satélites ganharam nova relevância. A disponibilização de imagens de forma gratuita, assim como a preocupação com a manutenção de séries históricas e o desenvolvimento de constelações que diminuam a resolução temporal acabaram possibilitando esse retorno do pixel como chave analítica central no sensoriamento remoto.
\par

\section{Técnicas e Algoritmos para a Análise Temporal de Áreas Florestadas}
Analisar áreas florestadas sob uma perspectiva temporal possibilita a identificação de processos e padrões que uma simples caracterização espectral mais tradicional não é possível de identificar devido a limitações ligadas a resolução espacial, radiométrica e espectral. Sendo assim, ao incluir a dimensão temporal, é possível entender dinâmicas como a supressão da floresta em um dado momento, assim como também os distúrbios naturais e degradações de origem antrópica ao longo do tempo.
A degradação associada ao desmatamento e posterior uso agrícola da terra, seguido do abandono e consequente retorno da vegetação através de processos de regeneração natural, ou então um processo de degradação mais lento, como a extração seletiva de madeira, são exemplos de mudanças no uso e cobertura da terra que só podem ser compreendidos através de técnicas como as que serão mostradas neste trabalho. 
\par
Técnicas de detecção de mudança possuem uma longa história na área de sensoriamento remoto. Desde as primeiras aplicações utilizando sensores TM do satélite Landsat 5 na década de 1980 e 1990, muitos estudos foram feitos. Inicialmente os estudos da área visavam somente na aplicação de técnicas mais tradicionais como o mapeamento das áreas de interesse utilizando técnicas tradicionais de classificação de imagens tanto de forma supervisionada como não supervisionada e posterior cálculo da diferença entre as duas ou mais imagens. Neste caso, somente métricas como o ganho e perda de área e sua consequente espacialização poderiam ser extraídas e visualizadas. No entanto, com o aumento do poder computacional e consequente amadurecimento das técnicas, softwares e bibliotecas disponíveis, o processamento de dados multi-temporais passaram a ser entendidos em sua totalidade. Ou seja, com a aplicação de técnicas menos reducionistas. O que isso significa na prática é que a análise de mudanças de áreas de floresta passou a ser feita através da manipulação tanto da criação de composições anuais, assim como intra-anuais e também da análise de toda a série temporal sem maiores cortes.
\par
Para o processamento de séries temporais utilizando imagens de satélite é necessário se preocupar também com o pré processamento das mesmas para que a aplicação do algoritmo, seja ele qual for, não influencie o resultado final com ruídos derivados de falta de calibração geométrica entre as imagens de diferentes datas, assim como a variação radiométrica, além da presença de possíveis sombras. Para isso, é necessário utilizar técnicas de pré-processamento como a ortorretificação e a correção radiométrica (atmosférica) das imagens. Dependendo do satélite utilizado é possível utilizar métodos implementados pela própria agência distribuidora, o que é bastante recomendado. No caso da correção geométrica das imagens, muitas já são disponibilizadas após serem tratadas automaticamente por método de correção subpixel ~\cite{Gutjahr2014}. Já para a correção radiométrica existem dois tipos: as calibração absoluta e a relativa. A absoluta faz a calibração transformando os valores digitais em valores físicos de superfície e aplicando algoritmos como o 6S ~\cite{Sagan2004}, já a calibração relativa utiliza uma imagem de referência como base e aproxima os valores do resto da série de acordo com a imagem base. Trabalhos comparando os dois métodos já foram desenvolvidos utilizando imagens Landsat e apresentam resultados similares ~\cite{Chen2010}. Produtos da série Landsat ainda possuem métodos internos de correção além de máscaras para a filtragem de nuvens, sombras e outras características das imagens ~\cite{ZHU2015269, ZHU201283, Huang2010}, o que facilita bastante a etapa de pré-processamento.
\par
Alguns algoritmos foram sendo desenvolvidos ao longo dos últimos anos com o objetivo de lidar melhor com essas séries. Um dos mais antigos é o BFAST \sloppy  ~\cite{VERBESSELT2010106, VERBESSELT20102970, VERBESSELT201298} que possui uma versão visando dados espaciais denominada bfastSpatial ~\cite{bfastSpatial}. As duas ferramentas possuem o mesmo algoritmo de detecção, mas a versão espacial se diferencia por conta de uma série de funções voltadas para facilitar o pré-processamento dos dados com o intuito de construir inicialmente a série temporal. A construção das séries temporais em ambiente offline sem ajuda de ferramentas mais modernas é, de fato, bastante trabalhosa e contém muitas etapas necessárias para que a análise final possa ser feita sem maiores problemas e sem a geração de ruídos por conta de dados de entrada problemáticos. É importante notar também que as duas ferramentas, diferentemente de outros algoritmos que serão analisados neste trabalho, não são voltados exclusivamente para a detecção de distúrbios em ambientes florestais,  mas sim em basicamente qualquer outro tipo de uso. No entanto, o uso das duas ferramentas em ambientes florestais é provavelmente o mais comum entre os trabalhos existentes. 
\par
O BFAST tem como ideia geral analisar a série temporal de imagens analisando os valores pixel à pixel e detectando quebras (\textit{breakpoints}) de pixels que tenham valor discrepante do valor médio esperado. É possível detectar mais de uma quebra em uma mesma série de pixels, o que é interessante principalmente para análise de usos agrícolas, assim como quando comparados a áreas florestais. O algoritmo ainda pode ser utilizado em objetos, como apresentado por Siti Latifah ~\cite{LATIFAH2016}, onde a integração com técnicas de GEOBIA apresentou ótimos resultados. Este resultado demonstra ainda um novo potencial a ser explorado, onde a integração de dois paradigmas (pixel/objeto) através do processamento de séries temporais pode trazer novas formas de análise e até mesmo a implementação de novos algoritmos.
\par
Já outros trabalhos utilizando o BFAST demonstraram a capacidade do mesmo na detecção de quebras para a caracterização de distúrbios cíclicos em florestas com o objetivo de explicar mudanças estruturais que acabam influenciando diretamente na qualidade da floresta presente e não só na sua simples presença ou não presença. ~\cite{JAKOVAC2017, DUTRIEUX2016112}.
\par
Além do BFAST, outras implementações computacionais foram desenvolvidas com o objetivo de analisar séries temporais para detecção de padrões em tipos variados de uso do solo. Este é o caso do Timesat ~\cite{Jnsson2004TIMESATA}. O Timesat, apesar de ter a aplicação mais voltada para a caracterização de tipos de culturas agrícolas através da interpretação da série, também possui uso na caracterização de tipos e também de distúrbios em florestas ~\cite{Wenbo2017}. O Timesat ainda possui diversas ferramentas internas para o tratamento de ruídos, tratamento de dados faltantes e composição de séries sazonais utilizando o algoritmos como o Savitzky-Golay ~\cite{Savitzky1964}.
\par
Ao aprofundarmos mais, para além das técnicas mais tradicionais, podemos entender que esses algoritmos apresentados, assim como muitos outros existentes se diferenciam entre si. Esses novos métodos de análise podem ser categorizados em quatro sub-categorias: algoritmos baseados na detecção de mudanças baseados em limiares, os baseados em ajude de curvas, os baseados no ajude de trajetórias e os baseados na segmentação de trajetórias ~\cite{Banskota2014, Hirschmugl2017}.

\subsection{Detecção de Mudanças Baseado em Limiares}
Os métodos de detecção baseados em limiares funcionam buscando a diferenciação de áreas de floresta e não floresta, e posteriormente separando áreas de floresta "intacta" das que sofreram algum tipo de distúrbio ou processo de degradação. A ideia é utilizar uma série temporal previamente tratada formadas tanto puramente por índices de vegetação, como pela integração de diversas bandas espectrais e bandas sintéticas derivadas de estatísticas da própria série. Esses métodos possuem um grande potencial e aplicação, mas ao mesmo tempo tem como ponto negativo a necessidade da delimitação empírica de limiares, o que dificulta bastante a replicabilidade dos trabalhos.

\subsection{Detecção de Mudanças Baseado no Ajuste de Curvas}
A utilização de métodos baseados no ajuste de curvas em áreas florestadas tem como objetivo entender o comportamento espectral primeiramente aplicando uma linha de regressão, onde dependendo da inclinação da mesma é possível detectar a presença ou ausência de mudanças significativas. Além disso, o sinal da inclinação determina também o ganho ou perda de biomassa. O lado negativo desse método é a necessidade da suposição de uma certa normalidade entre os dados de entrada, o que em sensoriamento remoto é quase sempre muito difícil de se obter. Isso já tende a limiar a aplicação de métodos como esse a sensores com menor resolução temporal como os presente no projeto MODIS (\textit{Moderate-Resolution Imaging Spectroradiometer}), ou então de composições muito bem estruturadas de sensores como o Landsat. A utilização desse método pode ser exemplificado pelo trabalho desenvolvido no bioma amazônico utilizando imagens MODIS onde é demonstrado a relação entre o corte seletivo e a mudança da resposta fenológica da vegetação na região ao longo do tempo ~\cite{KOLTUNOV20092431}.

\subsection{Detecção de Mudanças Baseado no Ajuste de Trajetórias}
As técnicas de ajuste de trajetórias se diferenciam dos anteriores por analisarem as mudanças a partir de trajetórias idealizadas. Através da aplicação de métodos comparativos (ajuste) entre a série estudada com a de referência, seja através do cálculo da distância euclidiana como pela utilização de métodos mais complexos como o DTW (\textit{Dynamic Time Warping}) ~\cite{VELICHKO1970223, SakoeChiba71, Berndt1994} e sua versão para a classificação de uso do solo, o TWDTW (\textit{Time-Weighted Dynamic Time Warping}) ~\cite{Maus2016, Maus2019}, o ajuste de trajetórias funciona como um método de análise supervisionado já que depende de amostras de "treino" para a obtenção de resultados satisfatórios.
\par
Este tipo de algoritmo, devido a sua natureza de carácter supervisionado, tem sido utilizado principalmente em estudos aplicados na tipificação de culturas agrícolas, onde a diferenciação dos alvos só pode ser feita utilizando conhecimentos relativos ao comportamento espectral do alvo no tempo. Este tipo de algoritmo necessita de uma densidade de imagens maior que as outras técnicas, já que é necessário o maior grau possível de precisão na série para encaixar e detectar as características entre as duas séries. Para este tipo de aplicação são utilizados normalmente imagens derivadas do sensor MODIS, devido a sua resolução temporal e as suas aplicações na caracterização do comportamento espectral de grandes áreas agrícolas, onde a resolução espacial do satélite (250m) não impede a análise de ser realizada. É possível utilizar imagens Landsat com este método, mas é necessário acumular uma grande densidade de imagens através da fusão do histórico de vários sensores e/ou trabalhando com área de interseção entre path/row diferentes, o que limita sua aplicação ~\cite{Bendini2016}. Além dessas limitações, outros desafios vem sendo enfrentados em relação a dificuldade de 
detecção de distúrbios pontuais de corte seletivo e posterior regeneração natural do local, já que amostras muito bem definidas para este tipo de distúrbio precisam ser coletadas. No entanto, exemplos de aplicação em florestas tropicais são presentes ~\cite{Hirschmugl2013, KENNEDY2007370}.


\subsection{Detecção de Mudanças Baseado na Segmentação de Trajetórias}
O método de segmentação de trajetórias pode ser exemplificado pelo algoritmo Landtrendr ~\cite{KENNEDY20102897, KENNEDY2012117}, que funciona dividindo a série temporal em segmentos para posterior estudo. Este tipo de abordagem favorece estudos onde em uma mesma série (pixel) é possível entender vários processos. É possível não só a detecção de distúrbios em períodos de tempo curtos (provável evento de desmatamento), assim como na detecção de distúrbios de longo prazo (possível processo de degradação/corte seletivo), como processos de regeneração com longa duração (regeneração natural) e de curto prazo (floresta plantada e projetos de reflorestamento). Esses segmentos são realizados a partir da estipulação de vértices durante a série. Os vértices representam os pontos da série onde houveram algum tipo de mudança na qual o algoritmo considerou relevante. A escolha da relevância para a criação de um vértice (quebra) é feita de acordo com regras pré-definidas pelo usuário. Outra vantagem desse método é a possibilidade de análise sem a necessidade de amostras de eventos exemplo como no caso dos algoritmos de ajustes de trajetórias.
No entanto, o método de segmentação de trajetórias também possui suas desvantagens. A principal é que o algoritmo desconsidera efeitos sazonais da vegetação. Além disso, é um método que apesar de possuir alguns trabalhos aplicando suas técnicas em florestas temperadas ~\cite{PFLUGMACHER2012146, Griffiths2015}, ainda possui poucos estudos em áreas tropicais.

\subsection{Exemplos de Ferramentas e suas Características}
Podemos listar brevemente alguns algoritmos de detecção automática de mudanças que apesar do nome, na verdade não realizam exatamente o trabalho de detecção da mudança, mas sim mais um processo de predição dessas mudanças, já que trabalham a partir de técnicas de regressão. Algumas características e particularidades de cada algoritmo também serão apresentadas. Todos foram desenvolvidos nos últimos anos e representam grande parte dos métodos de detecção automática de distúrbios presentes na literatura recente. São eles:

\begin{itemize}
  \item CCDC - \textit{Continuous Change Detection and Classification} ~\cite{ZHU2014152} - O CCDC apresenta funções de análise de séries temporais utilizando não composições anuais ou intra-anuais, mas sim toda a série de imagens de entrada, o que o difere da maioria dos algoritmos exemplificados aqui. O CCDC funciona apenas com imagens sem a presença de nuvens e sombra e busca encontrar padrões de sazonalidade, tendências e quebras na série. Uma característica interessante do CCDC é que o algoritmo é capaz de gerar imagens "sintéticas" para qualquer data presente na série de entrada ~\cite{ZHU201567}. Estas imagens sintéticas geradas são utilizadas obrigatoriamente como dado de entrada por algoritmos como o MIIICA e o ITRA e também podem ser utilizadas opcionalmente por algoritmos como o Landtrendr e o VCT, ao invés de utilizar as imagens originais com valores de reflectância da superfície.
  
  \item VCT - \textit{Vegetation Change Tracker} ~\cite{Huang2010, THOMAS201119} - O VCT utiliza composições sem nuvem de imagens com valor de reflectância da superfície, ou de composições sintéticas geradas pelo algoritmo CCDC e extrai uma métrica de similaridade a áreas de floresta intacta. O algoritmo prediz os distúrbios detectando padrões que se afastam da métrica de similaridade. 
  
  \item LandTrendr ~\cite{KENNEDY20102897, KENNEDY2012117} - O Landtrendr, desenvolvido pelo Environmental Monitoring, Analysis and Process Recognition Lab da Universidade de Oregon, também trabalha tanto com composições de imagens com valores de reflectância da superfície como com imagens sintéticas geradas pelo CCDC. O algoritmo necessita que as imagens não possuam interferência de nuvens e sombras e gera seus dados de saída através da aplicação da técnica de segmentação de trajetórias. O Landtrendr pode gerar saídas como métricas tanto para distúrbios de perda como de ganho, além de detectar se as mudança ocorreram de forma lenta ou rápida, possibilitando também o cálculo da duração dos eventos segmentados previamente gerando não só dados contínuos como a magnitude, assim como dados discretos como a durança e o ano da detecção. É certamente um dos algoritmos com maior quantidade de informação gerada por rodada, o que facilita em muito sua utilização. Outra vantagem do Landtrendr é que apesar de ter sido implementado inicialmente utilizando a linguagem de programação proprietária IDL em um ambiente bastante complicado de manuseio através do software ENVI, foi recentemente implementado diretamente na plataforma online Google Earth Engine ~\cite{GORELICK201718}, o que veio a facilitar e muito sua utilização pela comunidade ~\cite{Kennedy2018}. A conversa do algoritmo para a plataforma online do Google possibilitou ainda que o tempo de processamento do mesmo fosse reduzido significativamente. No entanto, a plataforma restringe o processamento para no máximo a área equivalente de uma imagem Landsat por vez.
  
  \item EWMACD - \textit{Exponentially Weighted Moving Average Change Detection} ~\cite{Brooks2014} - \lipsum[1]
  \item MIICA - \textit{Multi-index Integrated Change Analysis} ~\cite{JIN2013159} - \lipsum[1]
  \item VerDET - \textit{Vegetation Regeneration and Disturbance Estimates through Time} ~\cite{Hughes2017} - \lipsum[1]
  \item ITRA - \textit{Image Trends from Regression Analysis} ~\cite{VOGELMANN201292} - \lipsum[1]
  \item Shapes-NBR ~\cite{Meyer2013, Moisen2016} - \lipsum[1]
\end{itemize}
\lipsum[1]

\subsection{Validação de Séries Temporais}
\lipsum[1]
TimeSync ~\cite{COHEN20102911}

\begin{comment}
    Brazil made an ambitious pledge of 12 Mha as a contribution to the Bonn Challenge global target of bringing 150 and 350 Mha of  degraded/deforested lands under restoration by 2020 and 2030, respectively (www.bonnchallenge.org).
    
    It is also part of Brazil’s pledge to the Paris Climate Agreement 
    (http://www.mma.gov.br/images/arquivo/80108/BRASIL%20iNDC%20portugues%20F 
    INAL.pdf) and its National Policy for Native Vegetation Recovery 
    (http://www.mma.gov.br/images/arquivos/florestas/planaveg_plano_nacional_recuperacao_vegetacao_nativa.pdf).
\end{comment}


\bibliographystyle{apalike}
\bibliography{bibliography.bib}
\end{document}





