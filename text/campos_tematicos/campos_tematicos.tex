\documentclass{article}
\usepackage[portuguese]{babel}
\usepackage[utf8]{inputenc}
\usepackage{comment}

% Keywords command
\providecommand{\keywords}[1]
{
  \small	
  \textbf{\text{Palavras-chave---}} #1
}

\title{Perspectivas sobre a Utilização de Algoritmos para o Processamento de Séries Temporais no Estudo de Distúrbios Florestais}
\author{Eduardo Ribeiro Lacerda }
\date{\the\year}

\begin{document}
\maketitle

\begin{abstract}
O objetivo deste trabalho é apresentar o estado da arte sobre a aplicação de técnicas de análise e séries temporais em imagens de satélite orbitais para o estudo de áreas de floresta que sofreram ou não algum tipo de distúrbio.    
\end{abstract}

\vspace{1cm}
\keywords{Séries Temporais; Algoritmos; Landtrendr; Distúrbios}

\section{Introdução}
A aplicação de técnicas de análise de séries temporais para o monitoramento espaço-temporal da paisagem e em específico para o de áreas florestadas vem possibilitando o desenvolvimento de estudos que visam para além de uma descrição estática da paisagem através da elaboração de mapas de uso e cobertura, um entendimento mais aprofundado sobre os processos ocorridos na paisagem ao estabelecer o tempo como ator importante. A incorporação do tempo possui importância estratégia pois possibilita tipos de monitoramentos que se tornam cada vez mais importantes sob os aspectos políticos e sócio-econômicos, como os acordos internacionais de restauração florestal e políticas de REDD+ ~\cite{BOS2019295, CROUZEILLES2019}. 
\par
Além disso, analisar áreas florestadas sob uma perspectiva temporal possibilita a identificação de processos e padrões que uma simples caracterização espectral mais tradicional não é possível de identificar devido a limitações ligadas a resolução espacial, radiométrica e espectral. Sendo assim, ao incluir o a dimensão temporal, é possível entender dinâmicas como a supressão da floresta em um dado momento, como também os distúrbios naturais e degradações de origem antrópica ao longo do tempo.
A degradação associada ao desmatamento e posterior uso agrícola da terra, seguido do abandono e consequente retorno da vegetação através de processos de regeneração natural, ou então um processo de degradação mais lento, como a extração seletiva de madeira, são exemplos de mudanças no uso e cobertura da terra que só podem ser compreendidos através de técnicas como as que serão mostradas neste trabalho. \par
Técnicas de detecção de mudança possuem uma longa história na área de sensoriamento remoto. Desde as primeiras aplicações utilizando sensores TM do satélite Landsat 5 na década de 1980 e 1990, muitos estudos foram feitos. Inicialmente os estudos da área visavam somente na aplicação de técnicas mais tradicionais como o mapeamento das áreas de interesse utilizando técnicas tradicionais de classificação de imagens tanto de forma supervisionada como não supervisionada e posterior cálculo da diferença entre as duas ou mais imagens. Neste caso, somente métricas como o ganho e perda de área e sua consequente espacialização poderiam ser extraídas e visualizadas. No entanto, com o constante avanço da tecnologia computacional tanto no poder de processamento como no de armazenamento e no consequente amadurecimento das técnicas, softwares e bibliotecas disponíveis, o processamento de dados multi-temporais passaram a ser entendidos em sua totalidade. Ou seja, sem a necessidade prévia da aplicação de técnicas reducionistas. O que isso significa na prática é que a análise de mudanças de áreas de floresta passou a ser feita através da manipulação e criação de composições anuais ou intra-anuais ou então na pura análise de toda a série temporal sem maiores cortes. 
\par
A capacidade de análise utilizando algoritmos voltados para o processamento de séries temporais necessita, antes de tudo, da composição de uma série de dados com uma densidade mínima para que o processo que esteja sendo estudado possa ser identificado. No caso da utilização de imagens de satélite isso se torna um limitador importante, já que a disponibilidade de satélites imageadores historicamente nunca foi alta e tem custo extremamente elevado de produção quando comparado a outros tipos de sensores. Ou problema é que os satélites possuem um tempo de revisita que em muitos casos não possibilitam que uma série mais densa possa ser estruturada. Além disso, problemas como a presença de nuvens e sombras, assim como ruídos na própria imagem e a diferenciação espacial da área de imageamento do satélite devido a heterogeneidade da distribuição solar no planeta dificultam ainda mais esse processo. No entanto, com o tempo, outros satélites foram sendo desenvolvidos e propositalmente pensados com o objetivo de incorporar as séries de imagens já existentes derivadas de projetos antigos e abandonados com os recém lançados. Um exemplo disso foi o lançamento da série Sentinel 2 (A e B) pela Agência Espacial Europeia com uma resolução espectral e espacial similar as encontradas na série Landsat, o que possibilita uma fusão entre imagens de satélites diferentes com o objetivo de aumentar a densidade de imagens. 

\begin{comment}
    Brazil made an ambitious pledge of 12 Mha as a contribution to the Bonn Challenge global target of bringing 150 and 350 Mha of  degraded/deforested lands under restoration by 2020 and 2030, respectively (www.bonnchallenge.org).
    
    It is also part of Brazil’s pledge to the Paris Climate Agreement 
    (http://www.mma.gov.br/images/arquivo/80108/BRASIL%20iNDC%20portugues%20F 
    INAL.pdf) and its National Policy for Native Vegetation Recovery 
    (http://www.mma.gov.br/images/arquivos/florestas/planaveg_plano_nacional_recuperacao_vegetacao_nativa.pdf).
\end{comment}


\bibliographystyle{apalike}
\bibliography{bibliography.bib}
\end{document}





