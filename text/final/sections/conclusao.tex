\begin{titlepage}
    \centering
    \vspace*{\fill}

    \vspace*{0.5cm}

    \Large% \bfseries
    Conclusões

    \vspace*{5cm}

    % \large Eduardo Ribeiro Lacerda

    \vspace*{\fill}
\end{titlepage}

\section{Conclusões}

\hspace{13pt} A incorporação de novas tecnologias para a detecção de mudanças na cobertura vegetal como as aplicadas neste trabalho se mostrou satisfatória para sua aplicação em florestas tropicais. Mesmo ao utilizar o Landtrendr com um mesmo conjunto de parâmetros originalmente aplicados em florestas temperadas em fitofisionomias tropicais da Mata Atlântica, o algoritmo apresentou um resultado significativo. O uso do Google Earth Engine se mostrou essencial tanto para o processamento, quanto armazenamento e também visualização e apresentação dos resultados. Sem a possibilidade de processamento na plataforma, a aplicação de algoritmos de detecção de mudança em larga escala seria limitada a poucos centros de pesquisa e a projetos com verba elevada.

No entanto, algumas limitações também foram percebidas durante o processo de teste da solução. Apesar do ganho significativo na capacidade e tempo de processamento, o Landtrendr só consegue ser executado em imagens com área similar a uma cena Landsat por vez, o que torna necessário a estruturação de uma plano de execução e agregação posterior dos resultados de forma que os mesmos não sejam sobrepostos. Além disso, ainda existem poucos casos de aplicação da ferramenta, uma documentação em constante desenvolvimento e limitações em relação a aplicação do algoritmo utilizando índices ou bandas que ainda não foram implementados na ferramenta. A incorporação de novos índices ainda é feita pela equipe desenvolvedora à partir de pedidos formais por email ou pela sessão \textit{issues} do GitHub. 

Há também, pouca informação e pesquisa sobre os parâmetros adotados pela ferramenta, assim como sobre os melhores índices/bandas. A incorporação da banda extra inicialmente chamada DSNR, atualmente CSNR, possibilitou entender quais bandas podem obter o melhor resultado de acordo com a área de estudo, mas como visto neste trabalho, quando aplicada a áreas extensas, os resultados não se mostraram robustos o suficiente. Houve uma variação muito alta nos erros associados para que os mesmos pudessem ser aplicados e uma banda ou índice pudesse ser escolhido para a aplicação da ferramenta.

Contudo, mesmo utilizando os parâmetros padrão da ferramenta, assim como o uso do NDVI como índice base para sua aplicação, os resultados obtidos pela ferramenta foram surpreendentes. Os resultados mostraram uma acurácia de 90\% e kappa de 0.85 em áreas de floresta ombrófila densa,  88\% e kappa de 0.82 em florestas ombrófilas mistas, e acurácia de 84\% com kappa de 0.76 em áreas de vegetação estacional semidecidual. Com uma acurácia global de 86.7\% e kappa de 0.8 pode se dizer que o algoritmo apresentou um resultado excelente para uma região tão extensa e diversa como o bioma. O TimeSync, ferramenta desenvolvida e disponibilizada abertamente pelos mesmos criadores do algoritmo se mostrou uma ferramenta essencial para a etapa de validação, sendo também um ótimo recursos para a etapa de validação de outros projetos, mesmo nos que utilizem outros algoritmos de detecção. O TimeSync possui a limitação de trabalhar apenas com séries temporais anuais, o que para algoritmos similares como o CCDC que trabalha com séries inteiras sem divisão temporal fixa, é um problema. No entanto, é certamente uma ferramenta que pode inspirar o desenvolvimento de opções similares que possuam a possibilidade de trabalho com outros formatos de análise. O campo ainda é novo e muitos algoritmos foram desenvolvidos na última década. A incorporação dos mesmos em plataformas como o Google Earth Engine mostra o crescimento da popularidade, e resultados como o obtido no caso da Mata Atlântica mostram a maturidade, potencialidade e viabilidade das mesmas. 

Os resultados obtidos pelo Landtrendr para a Mata Atlântica, mostra um bioma que nos 33 anos (1985 - 2018) teve mais ganhos (11,667 km2) do que perdas (8,796 km2). Tanto os eventos de ganho quanto de perda, em sua maioria, aconteceram durante a década de 1990, o que refletiu diretamente nos resultados obtidos para as unidades de conservação do bioma. Uma alta dinâmica no uso e cobertura da terra devido a idade similar tanto dos eventos de ganho quanto perda foi detectada principalmente em estados com o do Paraná, Santa Catarina e Bahia. 

Os dados ainda mostraram eventos de ganho e perda, principalmente no sudeste, em partes mais altas e/ou com alto grau de declividade, o que demonstra nessas regiões com histórico de antropização mais forte, uma provável maior disputa por terras em regiões menos antropizadas. Já no sul, com exceção do Paraná, a maior parte das detecções aconteceram em áreas mais baixas, seguindo a mesma lógica presente no nordeste. Mudanças nesses locais aconteceram principalmente em remanescentes já fragmentados presentes nas áreas mais planas. Outro resultado significativo apresentado pelo algoritmo foi a alta quantidade de perdas com declividade considerada íngreme em todo o bioma, provável influência causada por movimentos de massa nessas regiões. 

Uma característica importante dos resultados obtidos por algoritmos de detecção de mudança baseado na análise de séries temporais, e principalmente no caso do Landtrendr, é a quantidade realmente significativa de resultados derivados que podem ser gerados. A execução da ferramenta resulta em ao menos seis camadas, que por si só já podem servir de apoio a análise, monitoramento e tomada de decisão estratégica em áreas cruciais para o desenvolvimento sustentável. A possibilidade de mudança de índice e dos parâmetros de execução possibilita a aplicação da ferramenta em projetos com diferentes objetivos e até mesmo da aplicação em áreas não tradicionais, como áreas urbanas.

O foco deste trabalho foi o teste do algoritmo Landtrendr em áreas tropicais com o objetivo do desenvolvimento de uma metodologia para aplicação da mesma em áreas maiores, visando uma maior diversidade de aplicações, inclusive em grandes projetos de monitoramento. Estamos na década da restauração, e o uso de soluções como a estudada durante este trabalho se mostrou robusta o suficiente para sua consideração em trabalhos futuros que tenham como ao menos um dos seus objetivos a incorporação do tempo e o maior entendimento dos processos da dinâmica da paisagem. Este entendimento se mostra cada vez mais essencial para que os projetos de conservação e restauração sejam analisados ou monitorados com alto nível de complexidade e maior qualidade. 

No início deste trabalho existiam pouquíssimas iniciativas de utilização do Landtrendr, assim como de outras ferramentas similares no território nacional. O Landtrendr possuía apenas sua versão clássica para ENVI/IDL e somente em 2018 que passou a contar com sua implementação para a plataforma do Google. Com isso, novas possibilidades surgiram, e o esforço foi para que este documento pudesse contribuir tanto para uma discussão mais ampla sobre a ferramenta, como para o uso de soluções similares e para as possibilidades de análises baseadas em séries temporais em um contexto territorial extenso como o nosso. Os resultados, como discutido anteriormente, foram satisfatórios e muitas possibilidades de análise não caberiam em um documento como este. No entanto, acredita-se que o mesmo pode contribuir tanto no sentido técnico voltado para a comunidade de geotecnologias e sensoriamento remoto, assim como para um maior entendimento das dinâmicas espaciais ocorridas no bioma em um período chave da democracia nacional.   