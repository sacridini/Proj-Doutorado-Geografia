\documentclass{article}
\usepackage[utf8]{inputenc}
\usepackage{graphicx}

\graphicspath{{images/}}

\title{Notas da Tese - Monitoramento de Distúrbios Florestais na Mata Atlântica Brasileira. Análise das dinâmicas ocorridas de 1985 à 2018.}
\author{Eduardo Ribeiro Lacerda }
\date{Criado em: Fevereiro 2019}

\begin{document}

\maketitle

\section{Ideias Sobre a Estrutura da Tese}
\begin{itemize}
    \item A tese deverá ter 3 capítulos.
    
    \item O primeiro capítulo deverá ser o elaborado para a matéria de campos temáticos que faz uma revisão sobre os métodos mais atuais de análise de séries temporais aplicadas ao estudo de degradação florestal.
    
    \item O segundo capítulo deverá ser sobre o Landtrendr em si. Deverá fazer uma breve explicação sobre as capacidades da ferramenta, apresentar a área de estudo, mostrar os parâmetros escolhidos (de preferência de forma referenciável), mostrar os resultados de acordo com os parâmetros escolhidos e finalmente os resultados com a validação no timesync.
    
    \item O terceiro capítulo provavelmente deve ser sobre as análises decorrentes da junção dos dois métodos de análise da paisagem. O capítulo deverá possuir análises pós cruzamento com base de dados consagradas sobre a mata atlântica e novas problematizações. Os dois primeiros capítulos servirão de base para o terceiro. 
    
    \item Será possível separar posteriormente o que é restauração e regeneração natural? (utilizar os dados do PACTO?)
\end{itemize}

\subsection{Estrutura da Tese}

\begin{itemize}
    \item Objetivos Gerais 
    
    \item Objetivos Específicos
    
    \item Material de Campos Temáticos
    
    \item Área de Estudo
    
    \item Texto entre cap.2 (random forest) e cap.3 explicando os resultados ruins da aplicação do random forest para o resto da mata atlântica. Citar o artigo da Hanna Meyer explicando a perda de qualidade da classificação no espaço. Seria necessário criar uma máscara muito mais complexa por conta da quantidade de amostras e complexidade do modelo.
    
\end{itemize}

\section{Possíveis perguntas a serem respondidas pós análise}
\begin{itemize}
    \item Como as unidades de preservação existentes da MA se comportaram ao longo dos 33 anos? (Parque Nacional da Serra da Bocaina, Parque Nacional da Serra dos Órgãos...)
    \item Como as APPs em especial se comportaram ao longo do mesmo período?
    \item Quais os anos com maior quantidade de supressão
    \item Quais os anos com maior quantidade de regeneração natural/restauração
    \item Analisar as diferenças entre perdas e ganhos de acordo com cada fitofisionomia
    \item Quais as áreas com maior taxa (rate) de mudança (tanto para perda como ganho)? 
    \item Análise de perdas e ganhos de acordo com os municípios
    \item Analisar as perdas e ganhos nos corredores ecológicos central e da serra do mar. 
\end{itemize}

\section{Objetivos}
\begin{itemize}
    \item Verificar se o Landtrendr consegue identificar com alta precisão mudanças ocorridas ao longo dos 34 anos da série.
    
    \item Verificar se o resultado da metodologia consegue boa performance mesmo considerando a aplicação do algoritmo em diferentes fitofisionomias utilizando um mesmo conjunto de parâmetros.
\end{itemize}

\section{Anotações Aleatórias}
\begin{itemize}
    \item Esse bioma possui 75,6\% das espécies ameaçadas e endêmicas do Brasil, o que o torna um dos mais prioritários para conservação no país. \textbf{Piratelli, Augusto João (2013). Conservação da biodiversidade: dos conceitos às ações.}
    \item Existem pelo menos 510 espécies em extinção, algumas em âmbito global, outras em âmbito nacional, e outras estão ameaçadas apenas no bioma: inúmeras espécies endêmicas como o pau-brasil e o mico-leão-preto acabam se tornando ameaçadas em todos os níveis desde o regional até o global. \textbf{ABARELLI, M.; et al. (2005). Espécies Ameaçadas e Planejamento da Conservação. Em Galindo-Leal, C.; Câmara, I.B. (Orgs). Mata Atlântica: Biodiversidade, ameaças e perspectivas}
\end{itemize}

\section{Anotações do Livro do Fabio Scarano}

Segundo SCARANO, 2014, na época da chegada dos portugueses no Brasil, a Mata Atlântica cobria cerca de 1,5 milhões de quilômetros quadrados, estendendo-se ao longo de 3 mil quilômetros da costa brasileira - do Rio Grande do Sul ao Rio Grande do Norte - e penetrando pelo interior, cruzando São Paulo, Minas Gerais e Mato Grosso do Sul até as fronteiras da Argentina e do Paraguai. Cerca de 500 anos depois, esse extenso e representativo bioma abriga mais de 100 milhões de pessoas, cerca de 1/4 das quais vive na pobreza. Aproximadamente 60\% das população brasileira vive na área coberta pela Mata Atlântica, o que em parte explica o fato de hoje apenas cerca de 12\% de sua cobertura natural ter persistido. (SCARANO, 2014).

"O mosaico de habitats e ecossistemas que compõe o domínio da Mata Atlântica abriga mais de 15.700 espécies de plantas e mais de 2.200 espécies de vertebrados registrados pela ciência (260-300 mamíferos; 930-990 aves; 200-300 répteis; 370-480 anfíbios; 300-350 peixes). AS estimativas variam, mas, de qualquer forma, impressiona que a Mata Atlântica, representando apenas 0.8\% da superfície terrestre do planeta, abrigue cerca de 5\% das espécies de vertebrados e 5\% da flora mundial. Mesmo tratando-se do bioma brasileiro mais estudado, ainda há muito por se descobrir. Entre 1990 e 2006, registraram-se mais de 1.190 novas espécies de plantas na Mata Atlântica, no coração da área urbana da cidade do Rio de Janeiro, novas espécies são encontradas com alguma frequência." (SCARANO, 2014)

"... parte expressiva da fauna e da flora da Mata Atlântica é endêmica, ou seja, não ocorre em nenhum outro lugar do planeta. Estima-se que entre 43\% e 45\% do total de espécies de plantas e vertebrados sejam restritas a esse bioma..." (SCARANO, 2014)


\section{Anotações sobre o Landtrendr}
Nem todo perda (loss greatest) registrada em um pixel significa que a mesma representa o menor valor de NDVI em toda a série. É possível que existam outros pontos do tempo onde o valor foi ainda menor:
\begin{figure}[htp]
    \centering
    \includegraphics[width=1.1\textwidth]{ltr_loss01.png}
    \caption{O valor de biomassa chegou a ser ainda menor do que quando a maior perda foi registrada.}
    \label{fig:galaxy}
\end{figure}

\end{document}
